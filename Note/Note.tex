\documentclass[a4paper,10pt]{article}
\usepackage[utf8]{inputenc}
\usepackage{amsmath}
\usepackage{amssymb}
\usepackage{graphicx}
\usepackage{booktabs}
\usepackage{floatflt}
\usepackage{verbatim}
\usepackage{caption}
\usepackage{subcaption}
\usepackage{morefloats}
\usepackage{braket}
\usepackage{hyperref}
\usepackage{siunitx}
\usepackage{bm}

\usepackage{adjustbox}
\usepackage{graphicx}


\begin{document}

% Title Page
\title{Spectral Density extraction}
\author{Manuel Naviglio}
\date{}
\maketitle

Let us consider an Euclidean time-ordered correlator at discrete values of the space-time coordinates and on a finite volume
\begin{equation}
C(t) = \frac{1}{V}\sum_{\vec{x}} \langle 0 | T( \mathcal{O}(t,\vec{x}) \mathcal{O}(0)) |0\rangle.
\end{equation}
If we consider the case $t>0$ and we insert a complete set of states we obtain
\begin{equation}\begin{split}
C(t)_{t>0} &= \frac{1}{V}\sum_{\vec{x}} \langle 0 |  \mathcal{O}(t,\vec{x}) \mathcal{O}(0)|0\rangle  = \frac{1}{V}\sum_{\vec{x}} \sum_n \langle 0 |  \mathcal{O}(t,\vec{x}) |n\rangle \langle n | \mathcal{O}(0) |0\rangle \\& = \frac{1}{V} \sum_n  e^{-E_n t} \sum_{\vec{x}} \langle 0 |  \mathcal{O}(t,\vec{x}) |n\rangle \langle n | \mathcal{O}(0) |0\rangle = \int_0^\infty d \omega \rho(\omega) e^{-\omega t}
\end{split}\end{equation}
where 
\begin{equation}
\rho(\omega) = \sum_n \delta(\omega- E_n) \sum_{\vec{x}} \langle 0 |  \mathcal{O}(t,\vec{x}) |n\rangle \langle n | \mathcal{O}(0) |0\rangle, 
\end{equation}
is a distribution called spectral density having support in correspondence of the states spectrum specified by the quantum numbers of the operator $\mathcal{O}$. If we also consider the other ordering of time, a further piece appears and the correlator can be written as
\begin{equation}\label{eq: CorrBasis}
C(t) = \int_0^\infty \rho(\omega) (e^{-\omega t} + e^{-\omega(T-t)}) = \int_0^\infty \rho(\omega) K_T(\omega, t),
\end{equation}
where $K_T(\omega, t)$ is the so called basis function. If we assume that the time extent of the lattice is infinite, then the basis function simply reduces to
\begin{equation}
K_\infty (\omega, t) = e^{-\omega t}.
\end{equation} 
In general, it's not easy to extract information about the spectral density. LQCD allows us to compute the l.h.s. of Eq. \eqref{eq: CorrBasis}, namely the values of correlator, for different discrete values of the Euclidean time, whose number depends on the time extension of the lattice we are considering. Using these inputs, we can then extract information about the spectral density making numerically a Laplace inverse transform. However, this is not possible when the used data are affected by uncertainties, as it's the case of the values of the correlator computed on the lattice. \\
In literature we can find principally three method that can be used to avoid this problems, namely Backus-Gilbert Method \cite{BG}, Tikhonov Regularization Method \cite{TR}, and a recent one proposed in \cite{Nazario}.\\
In the following, we are going to describe the three methods applying them to different 


\section{Backus-Gilbert Method}
\section{Hansen-Lupo-Tantalo Method}
\section{Tikhonov Regularization Method}


\newpage
\begin{thebibliography}{9}
\bibitem{BG}
G. Backus and F. Gilbert, Geophysical Journal Interna- tional 16, 169 (1968).

\bibitem{TR}
A. N. Tikhonov, Soviet Math. Dokl. 4, 1035 (1963).

\bibitem{Nazario}
M. Hansen, A. Lupo, and N. Tantalo, Phys. Rev. D99,
094508 (2019), arXiv:1903.06476 [hep-lat].

\end{thebibliography}



\end{document}



